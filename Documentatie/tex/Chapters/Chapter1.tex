% Chapter 1

\chapter{Introducere} % Main chapter title

\label{Chapter1} % For referencing the chapter elsewhere, use \ref{Chapter1} 

\lhead{Capitolul 1. \emph{Teoria compilarii}} % This is for the header on each page - perhaps a shortened title

%----------------------------------------------------------------------------------------

\section{De ce teoria compilarii}
	Prima oara cand am auzit despre ce inseamna un compilator a fost in liceu cand am invatat despre
	limbajul de programare Pascal.De atuncti pana in prezent am invatat o multime de alte limbaje.Cu 
	toate acestea nu stim cu adevarat ce inseamna un limbaj de programare desi pe unele dintre ele
	stiam sa le folosesc.Desi aveam cunostinte de limbaje formale tot nu puteam sa imi dau seama cum
	sunt imbinate acele modele matematice (gramatici,automatele finite) intr-o aplicatie care sa 
	generez cod.
	
		
	
	

%----------------------------------------------------------------------------------------


